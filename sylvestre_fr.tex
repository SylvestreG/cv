%%
%% Copyright (c) 2015 Sylvestre Gallon <ccna.syl@gmail.com>
%%
%% Permission to use, copy, modify, and distribute this software for any
%% purpose with or without fee is hereby granted, provided that the above
%% copyright notice and this permission notice appear in all copies.
%%
%% THE SOFTWARE IS PROVIDED "AS IS" AND THE AUTHOR DISCLAIMS ALL WARRANTIES
%% WITH REGARD TO THIS SOFTWARE INCLUDING ALL IMPLIED WARRANTIES OF
%% MERCHANTABILITY AND FITNESS. IN NO EVENT SHALL THE AUTHOR BE LIABLE FOR
%% ANY SPECIAL, DIRECT, INDIRECT, OR CONSEQUENTIAL DAMAGES OR ANY DAMAGES
%% WHATSOEVER RESULTING FROM LOSS OF USE, DATA OR PROFITS, WHETHER IN AN
%% ACTION OF CONTRACT, NEGLIGENCE OR OTHER TORTIOUS ACTION, ARISING OUT OF
%% OR IN CONNECTION WITH THE USE OR PERFORMANCE OF THIS SOFTWARE.
%%

\documentclass[11pt,a4paper,sans]{moderncv}
\moderncvstyle{casual}
\moderncvcolor{blue}
\usepackage[scale=0.75]{geometry}

\name{Sylvestre}{Gallon}
\title{Ing\'enieur R\&D}
\address{9 rue du Bas Vernay}{Saint Just Malmont 43240}{France}
\phone[mobile]{06~63~90~35~21}
\email{ccna.syl@gmail.com}
\social[linkedin]{sylvestre}
\social[viadeo]{sylvestre.gallon}
\social[github]{SylvestreG}
\photo[64pt][0.4pt]{sylvestre.jpg}
\extrainfo{31 ans - 13/10/1984 - Mari\'e}

\usepackage{multibib}
\newcites{book,misc}{{Books},{Others}}

\begin{document}

\makecvtitle

\section{Exp\'erience}

\cventry{2011--2016\\\includegraphics[scale=0.10]{jdsu.png}}
{Ing\'enieur R\&D}
{JDSU - VIAVI}
{Saint Etienne}
{}
{JDS Uniphase Corporation est une enterprise qui industrialise des produits
dans le domaine des communications r\'eseaux, du test des moyens de
communications, des lasers, et des solutions optiques.
J'ai rejoins JDSU au sein de l'equipe R\&D pour travailler sur des besoins
drivers. J'ai, lors de ces ann\'ees, travaill\'e \`a la refonte de la couche
r\'eseaux de leurs produits, ajout\'e des fonctionnalit\'es sans fil
(Wireless, Bluetooth) et je les ai aid\'e \`a prendre le virage du Cloud.
J'ai aussi activement particip\'e \`a la refonte de nos interfaces en poussant
\`a l'utilisation des libraries Qt \`a la place des librairies X11.
Lors de cette exp\'erience, j'ai developp\'e du code m\'etier en C C++ sous
Linux, mais aussi travaill\'e sur Yocto, UBoot, et le Kernel Linux.
}

\cventry{2010\\7 mois\\\includegraphics[scale=0.20]{systar.png}}
{Ing\'enieur R\&D}
{Systar}
{Lyon/Paris}
{}
{Systar est un \'editeur logiciels de gestion de performance 
op\'erationnelle permettant aux grandes entreprises d'optimiser l'efficacit\'e
de leurs activit\'es et de leur infrastructure informatique. 
Je suis rentr\'e chez Systar au sein de l'\'equipe de R\&D pour travailler sur 
l'architecture de leur future solution de capacity management. J'ai donc
travaill\'e \`a la conception d'un agent syst\`eme multi-plateforme.
}

\cventry{2007--2010\\\includegraphics[scale=0.15]{adeneo.jpg}}
{Ing\'enieur R\&D}
{Adeneo Embedded}
{Lyon}
{}
{Adeneo Embedded est une soci\'et\'e experte en logiciel embarqu\'e 
avec comme sp\'ecialit\'e le developement de BSP (Windows CE, Linux et
Android).
Mon travail a consist\'e \`a d\'evelopper diff\'erents drivers et diff\'erents
firmwares. J'ai travaill\'e pour eux sur de nombreuses architectures 
(Atmel, Freescale, NXP, Texas Instruments,...).
J'ai aussi r\'ealis\'e des missions d'expertise Windows CE pour des 
grands comptes comme Ingenico \`a Valence ainsi que les d\'eveloppements suivants:
\begin{itemize}
\item Recherche et d\'eveloppment sur les stacks et drivers USB.
\item D\'eveloppement du BSP neocore926.
\item D\'eveloppement du BSP at91sam9rlek.
\item D\'eveloppement de BootStrap et Bootloader.
\item Recherche et d\'eveloppment sur windows embedded compact chelan (ce7).
\item Recherche et d\'eveloppment sur des m\'ecanismes d'abstraction des SoC pour nos BSP.
\item Recherche et d\'eveloppment sur l'impl\'ementation d'un arbre de clock pour nos BSP.
\end{itemize}
}

\cventry{2007\\6 mois\\\includegraphics[scale=0.15]{lp-digital.png}}
{Responsable Informatique}
{LP Digital}
{Paris}
{}
{J'ai endoss\'e le r\^ole de responsable informatique pour une web agency.\newline{}
Mon travail a consist\'e \`a la gestion de nos 10 serveurs internes et 20 serveurs
externes (beaucoup de RedHat/Debian quelques 2003 server et OpenBSD). J'ai aussi
engag\'e une refonte de l'architecture IT.
J'ai eu \`a g\'erer les achats IT (serveurs, licences, mat\'eriel) et \`a venir en aide
aux utilisateurs du parc informatique.
}

\section{\'Education}

\cventry{2005--2009\\\includegraphics[scale=0.05]{epitech.png}}
{Master}
{Epitech}
{Paris}
{\textit{Promotion 2009}}
{Ecole Pour l'Informatique et les nouvelles TECHnologies}

\cventry{2003--2005}
{BTS}
{I.G.}
{Clermont-Ferrand}
{\textit{Promotion 2005}}
{Bts Informatique de Gestion Option D\'{e}veloppeur d'applications}

\section{OpenSource}

\cventry{2013--2014\\\includegraphics[scale=0.10]{openbsd.png}}
{Hacker}
{OpenBSD}
{Remote}
{}
{
Ma premi\`ere contribution au projet OpenBSD fut l'impl\'ementation du filesystem
FUSE.\newline{}
Je suis assez rapidement devenu mainteneur de certains paquets (fuse, 
openocd, nasm, tuxracer, etc...) et j'ai fait de nombreux commits sur l'architecture
ARM ( en particulier sur des cartes OMAP, SUNXI et Freescale...) 
}

\cventry{2009\\6 mois\\\includegraphics[scale=0.08]{freebsd.png}
\\\includegraphics[scale=0.05]{google.png}}
{FreeBSD Hacker pour le GSOC 2009}
{Google}
{Remote}
{}
{
J'ai particip\'e au Google Summer Of Code 2009. Ma r\'ealisation
durant ce Summer of Code, a \'et\'e de rajouter les supports de la libusb-1.0 
pour FreeBSD. J'ai aussi implement\'e des drivers USB et apport\'e certains
changements dans la stack USB d'Hans Petter Selasky.
}

\section{Comp\'etences techniques}
\cvdoubleitem{Langages}{C, C++\\Python, Perl, Go}{Syst\`emes}{Linux, OpenBSD, FreeBSD\\Windows, Windows Servers}
\cvdoubleitem{Assembleurs}{Avr, x86, Arm}{SoC}{IMX.6, OMAP4, AT91, ZYNQ}
\cvdoubleitem{Api}{Qt, DBus, Stl, GLib}{Bus}{USB, I2C, SPI, MMC, CAN}
\cvdoubleitem{Scripts}{Bash, ksh, tcsh}{Bootloader}{UBoot, Eboot, RedBoot}
\cvdoubleitem{Outils}{Git, Mercurial, svn, cvs}{EDI}{Eclipse, Vim, VisualStudio}
\cvdoubleitem{BDD}{sqlite, Postgresql, mysql}{IOT}{SigFox, Bluetooth, Wifi}
\cvdoubleitem{Embarqu\'e}{Yocto, LTIB, Buildroot}{Toolchains}{LLVM/Clang, gcc, binutils}

\section{Langues}
\cvitemwithcomment{Anglais}{Courant - Dans un souci de perfectionnement oral, je suis
actuellement}{}
\cvitemwithcomment{}{des cours hebdomadaires bas\'es sur le volontariat.}{}


\section{Centres d'int\^erets}
\cvitem{Sports M\'ecaniques}{Moto et voiture avec une passion pour les v\'ehicules anglais.}
\cvitem{Bricolage}{Je suis en train de r\'enover une maison sur Issoire.}

\end{document}
